\documentclass[16pt]{article}
\usepackage[margin=3cm]{geometry}
\setlength\parindent{0pt}

\usepackage{fancyhdr}
\fancypagestyle{plain}{
	\renewcommand{\headrulewidth}{0pt}
	\renewcommand{\footrulewidth}{0pt}
	\fancyhf{}
	\lhead{Use Case \#2}
	\chead{Comp 2005}
	\rhead{Project Group \#4}
	\cfoot{Page \thepage}
}
\pagestyle{plain}

\begin{document}
	\date{}
	\author{}
	\title{ \bf
		\underline{Fully Dressed Use Case \#2} \\ 
		\underline{(Toggle Colorblind Mode)}
	}
	\maketitle

	\textbf{Primary Actor:} Player \\

	\textbf{Stakeholders and Interests:}
	\begin{itemize}
		\item \underline{Player:}
		wants to be able to set the display to accommodate color deficient vision.
	\end{itemize}

	\textbf{Preconditions:}
	\begin{itemize}
		\item Player has opened up a menu (either the main menu, or the in-game menu)
	\end{itemize}

	\textbf{Postconditions:}
	\begin{itemize}
		\item The existence of shapes placed on each game piece (unique to the color of those pieces) will toggle.
	\end{itemize}

	\textbf{Main Success Scenario:}
	\begin{enumerate}
		\item The player selects "Toggle Colorblind Mode" from the menu.
		\item The system Will change the status indicator beside that selection to indicate the new status of the Colorblind Mode. 
		\item The system will change the display of any existing tiles to show/hide the assisting shapes.
		\item The system will display any subsequent tiles placed in the same manner.
	\end{enumerate}

	\textbf{Alternative Flows:} \\
	\underline{Alt3: No tiles have been placed yet}
	\begin{enumerate}
		\item The system makes no change to the display. Flow resumes at Main Success Scenario Step 4.
	\end{enumerate}

	\textbf{Special Requirements:}
	\begin{itemize}
		\item Shapes chosen to replace color as a tile's identifier must be easily distinguishable.
		\item It should be obvious which shape identifier belongs to which player. 
	\end{itemize}

	\textbf{Open Issues:}
	\begin{itemize}
		\item Should this status be set on a per-player basis?
	\end{itemize}
\end{document}